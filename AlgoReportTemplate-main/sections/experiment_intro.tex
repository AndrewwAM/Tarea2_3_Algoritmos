
La realización de los experimentos detallados en esta sección fue llevada a cabo bajo las siguientes condiciones:

\begin{itemize}
    \item Hardware:
    \subitem CPU: Ryzen 5 1600, 3.2GHz
    \subitem RAM: 24GB DDR4 2400MHz
    \subitem Almacenamiento: SSD 512GB
    \item Software:
    \subitem {SO: Nobara Linux 40 (KDE Plasma) x86$\_$64}
    \subitem Compilador: g++ 14.2.1
    \subitem Python: 3.12.6
    \subitem Librerías de Python:
    \subsubitem Pandas, Seaborn, matplotlib, Pathlib, Math, Numpy

\end{itemize}

Previo a la realización de las pruebas se cumplieron todos los requisitos para el correcto funcionamiento de los programas 
explicado en la sección anterior. Para ello se poblaron los archivos de costos de operación mediante \textbf{generateCosts.py}.

El problema que buscamos resolver corresponde a la \textit{Distancia de Damerau-Levenshtein} [cita], la cual es frecuentemente usada 
para resolver problemas como el \textit{funcionamiento del autocorrector de un teclado} [cita]. Por ello, para acercarnos a un escenario 
real las tablas de costo se poblaron sus valores asociados según que tan comunes son en el diccionario inglés y que tan probable es un error de typeo 
en función de la distancia entre caracteres en un teclado QWERTY. Esta forma de ajustar los costos determina la forma en la que se reconstruye la 
solución óptima, más no afecta los tiempos de ejecución ni el uso de memoria, los que solo dependen del algoritmo ejecutado y 
las características de los strings de entrada, las cuales dependen de cada dataset. Estos últimos fueron generados mediante \textbf{generateDatasets.py}. 
Las características de cada uno de los datasets se detallan en la siguiente sección.

Cada prueba se llevó a cabo utilizando el programa principal. Cada uno de los casos de prueba se ejecutó 4 veces, midiendo el tiempo de ejecución, el uso de memoria y guardando el resultado. 
Las últimas 3 fueron tomadas en cuenta a la hora de promediar para calcular los valores finales. La primera usa para realizar cualquier tipo de carga de caché que vaya a realizar el programa.

\pagebreak