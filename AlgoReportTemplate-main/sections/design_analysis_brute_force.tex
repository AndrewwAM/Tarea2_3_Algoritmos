%\epigraph{\textit{``Indeed, brute force is a perfectly good technique in many cases; the real question is, can we use brute force in such a way that we avoid the worst-case behavior?''}}{--- \citeauthor{taocv3}, \citeyear{taocv3} \cite{taocv3}}

\subsubsection{Descripción de la solución}
Resolver el problema de distancia de edición utilizando el enfoque de fuerza bruta no tiene gran complejidad en su diseño.
Buscamos igualar s1 a s2 probando todas las combinaciones posibles que surgen a partir de las 4 operaciones que dicta el problema.
Para este ejemplo se compara recursivamente y de manera ascendente los elementos de s1 y s2 con ayuda de índices ($i$ y $j$).
Por cada par de caracteres ubicados en s1[i] y s2[j], se evaluan en casos separados las cuatro acciones posibles a realizar
con ambos elementos, agregando el costo relacionado con la operación. Cada caso representaría una \textit{``ruta''} por la cual se buscaría llegar a la solución.
Luego de calcular cada uno de los casos, se escoje el de menor costo, retornando el costo óptimo de cada subtarea y, finalmente, el costo óptimo del problema original.
\subsubsection{Algoritmo utilizando fuerza bruta}
\begin{algorithm}[H]
    \SetKwProg{myproc}{Procedure}{}{}
    \SetKwFunction{BruteForce}{bruteForce}
    \SetKwFunction{CostInsert}{costInsert}
    \SetKwFunction{CostDelete}{costDelete}
    \SetKwFunction{CostSubstitute}{costSubstitute}
    \SetKwFunction{CostTranspose}{costTranspose}
    
    \DontPrintSemicolon
    \footnotesize

    \myproc{\BruteForce{s1, s2, i, j}}{
        \uIf{i = longitud(s1) \textbf{ and } j = longitud(s2)}{
            \Return 0  \tcp*[r]{Ambas cadenas están completas}
        }
        \uElseIf{i = longitud(s1)}{
            \Return (\CostInsert{s2[j]} + \BruteForce{s1, s2, i, j + 1})  \tcp*[r]{Insertar restante de s2}
        }
        \uElseIf{j = longitud(s2)}{
            \Return (\CostDelete{s1[i]} + \BruteForce{s1, s2, i + 1, j})  \tcp*[r]{Eliminar restante de s1}
        }
        
        \tcp{Calcula el costo mínimo entre las operaciones posibles}
        $costo_{del} \leftarrow \CostDelete{s1[i]} + \BruteForce{s1, s2, i + 1, j}$\;
        $costo_{ins} \leftarrow \CostInsert{s2[j]} + \BruteForce{s1, s2, i, j + 1}$\;
        $costo_{sub} \leftarrow \CostSubstitute{s1[i], s2[j]} + \BruteForce{s1, s2, i + 1, j + 1}$\;
        
        $costo_{trans} \leftarrow \infty$\;
        \uIf{i + 1 $<$ largo s1 \textbf{ and } j + 1 $<$ largo s2 \textbf{ and } s1[i] = s2[j+1] \textbf{ and } s1[i+1] = s2[j]}{
            $costo_{trans} \leftarrow \CostTranspose{s1[i], s1[i+1]} + \BruteForce{s1, s2, i + 2, j + 2}$\;
            }
            
        \Return $\min(costo_{del}, costo_{ins}, costo_{sub}, costo_{trans})$\;
    }

    \myproc{\CostInsert{b}}{
        \Return costo de inserción para `$b$` \tcp*[r]{Función de costo de inserción}
    }
    \myproc{\CostDelete{a}}{
        \Return costo de eliminación para `$a$` \tcp*[r]{Función de costo de eliminación}
    }
    \myproc{\CostSubstitute{a, b}}{
        \Return costo de sustitución entre `$a$` y `$b$` \tcp*[r]{Función de costo de sustitución}
    }
    \myproc{\CostTranspose{a, b}}{
        \Return costo de transposición entre `$a$` y `$b$` \tcp*[r]{Función de costo de transposición}
    }
    \caption{Algoritmo de Fuerza Bruta para calcular la distancia mínima de edición}
    \label{alg:bruteforce_edit_distance}
\end{algorithm}


\subsubsection{Ejemplo de ejecución}
Vamos a simular la ejecución del algoritmo diseñado con los datos de entrada s1: ``dog'' y s2: ``bag''.

\textbf{Paso 1:}:

// Se checkea si es un caso base

\begin{tabbing}
    $costo_{del} \leftarrow \CostDelete{s1[i]} + \BruteForce{s1, s2, i + 1, j}$\;


    \textbf{Vuelve al paso 1 con otros parámetros}:
\end{tabbing}

\begin{tabbing}
    $costo_{ins} \leftarrow \CostInsert{s2[j]} + \BruteForce{s1, s2, i, j + 1}$\;


    \textbf{Vuelve al paso 1 con otros parámetros}:
\end{tabbing}

\begin{tabbing}
    $costo_{sub} \leftarrow \CostSubstitute{s1[i], s2[j]} + \BruteForce{s1, s2, i + 1, j + 1}$\;


    \textbf{Vuelve al paso 1 con otros parámetros}:
\end{tabbing}

// Solo si es que se cumple la condición
\begin{tabbing}
    $costo_{trans} \leftarrow \CostTranspose{s1[i], s1[i+1]} + \BruteForce{s1, s2, i + 2, j + 2}$\;


    \textbf{Vuelve al paso 1 con otros parámetros}:
\end{tabbing}

\textbf{Paso 2:}

Al finalizar todas las llamadas recursivas se evaluan los resultados con:

\begin{tabbing}
    $\min(costo_{del}, costo_{ins}, costo_{sub}, costo_{trans})$
\end{tabbing}

y se retorna a la llamada anterior, para que esta \textbf{repita el paso 2} o en su defecto, retorne el valor a la llamada original, resolviendo el problema.


\subsubsection{Complejidad temporal y espacial}
En cada instancia de este algoritmo, este realiza al menos 3 llamadas recursivas a si mismo. La cuarta llamada ocurre en casos donde sea posible llevar a cabo esta operación, 
por ende, tomando en cuenta esta situación como el peor caso, el algoritmo tendría 4 llamadas recursivas. 

En caso de que un string sea mas largo que el otro, se llega a un caso base que, si bien tambien ejecuta una llamada
recursiva, esta solo realizaría acciones de complejidad lineal, por lo que las llamadas recursivas masivas del algoritmo 
\textbf{se detienen al alcanzar el tamaño del string más corto.} 

Si tuvieramos que combinar $n$ elementos de 4 maneras distinas, tendríamos $4^n$ combinaciones, siguiendo
este ejemplo y ya que este algoritmo evalúa todas las posibles combinaciones hasta el tamaño del string más pequeño, el enfoque fuerza bruta tiene 
una complejidad temporal de $O(4^{\min(n,m)})$, con $n$ y $m$ los largos de s1 y s2, respectivamente. 

En cuanto a la complejidad espacial, este algoritmo no ocupa ningún tipo de memoria adicional. Al ser in-place este tiene una complejidad espacial
de $O(1)$.


\subsubsection{Inclusión de Transposición y costos variables}
La inclusión de la acción de transposición al problema original añade un grado mayor de complejidad al problema. Esto aumenta la cantidad de llamadas
recursivas, empeorando considerablemente el rendimiento con respecto al problema sin esta opción. Los casos que presenten un mayor impacto debido a esta 
complejidad extra son aquellos donde la acción de ``transponer'' sea posible para todo el string, forzando que en cada llamada al algoritmo se ejecuten 
los 4 casos recursivos.

Con respecto a los costos variables, estos no afectan directamente la complejidad, ya que cada valor se recuperaría en tiempo constante a través de las funciones 
auxiliares.